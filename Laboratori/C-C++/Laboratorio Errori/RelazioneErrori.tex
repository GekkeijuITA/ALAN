\documentclass[10pt]{article}
\usepackage[italian]{babel}
\usepackage{geometry}
\usepackage{amsmath}
\usepackage{amssymb}
\usepackage{array}
\usepackage{tikz}


\title{Laboratorio Errori}
\author{Samuele Barrago [5703117], Daniele Sacco [5616921], Lorenzo Livio Vaccarecci [5462843]}
\geometry{margin=2cm}
\date{}

\begin{document}
\maketitle
\section{Primo esercizio}
\begin{quote}
    In questo esercizio vogliamo calcolare tramite la matricola del primo componente in ordine alfabetico, due semplici operazioni aritmetiche:
    \(a+(b+c)\) e \((a+b)+c\) utilizzando variabili di tipo double.
    \subsection{Calcolo di \(a, b, c\)}
    \begin{quote}
        Usando la matricola 5703117, indichiamo con \(d_{0}\) l'ultima cifra della matricola e con \(d_{1}\) la penultima.
        \begin{flalign*}
            a &= (d_{0} + 1) \cdot 10^i \text{ con } i = 0, 1, \ldots, 6 \\
            b &= (d_{1} + 1) \cdot 10^{20} \\
            c &= -b
        \end{flalign*}
    \end{quote}
    \subsection{Analisi dei risultati}
    \begin{center}
        \begin{tabular}{| c | c | c | c | c | c |}
            \hline
            \textbf{i} & \textbf{a} & \textbf{b} & \textbf{c} & \textbf{(a+b)+c} & \textbf{a+(b+c)} \\
            \hline
            0 &$6\cdot 10^{0}$  & $8 \cdot 10^{20}$ & $-8 \cdot 10^{20}$ & 0 & 6\\
            \hline
            1 &$6\cdot 10^{1}$ & $8 \cdot 10^{20}$ & $-8 \cdot 10^{20}$ & 0 & 60\\
            \hline
            2 &$6\cdot 10^{2}$ & $8 \cdot 10^{20}$ & $-8 \cdot 10^{20}$ & 0 & 600\\
            \hline
            3 &$6\cdot 10^{3}$ & $8 \cdot 10^{20}$ & $-8 \cdot 10^{20}$ & 0 & 6000\\
            \hline
            4 &$6\cdot 10^{4}$ & $8 \cdot 10^{20}$ & $-8 \cdot 10^{20}$ & 0 & 60000\\
            \hline
            5 &$6\cdot 10^{5}$ & $8 \cdot 10^{20}$ & $-8 \cdot 10^{20}$ & 655360 & 600000\\
            \hline
            6 &$6\cdot 10^{6}$ & $8 \cdot 10^{20}$ & $-8 \cdot 10^{20}$ & $6.03\cdot 10^{6}$ & $6\cdot 10^{6}$ \\
            \hline
        \end{tabular}
    \end{center}
    Sappiamo che per il \texttt{double} di C/C++ la precisione è
    \begin{equation*}
        u=\frac{1}{2}\cdot\left(B^{-t+1}\right) = \frac{1}{2}\cdot\left(2^{-51}\right) \simeq 2.22044605\cdot 10^{-16}  
    \end{equation*}
    Quindi possiamo dire che $a$ per essere "sentita" nelle operazioni deve essere \textbf{almeno} $u\cdot b\simeq 177635.684$ infatti fino all'iterazione $i=4$, $a$ non viene usata nel calcolo restituendo quindi come risultato 0. Per quanto riguarda l'iterazione $i=5$, 
\end{quote}
\newpage
\section{Secondo esercizio}
\begin{quote}
    Consideriamo come valore "corretto" il valore restituito dalla funzione \texttt{exp()} della libreria standard di C++.
    \begin{itemize}
        \item Errore assoluto: $|Taylor-exp|$
        \item Errore relativo: $\left |\frac{Taylor-exp}{exp}\right |$
    \end{itemize}
    \begin{tabular}{| c | c | c | c | c | c | c |}
        \hline
        \textbf{Alg} & \textbf{$x$} & \textbf{$N$} & \textbf{Taylor} & \textbf{exp()} & \textbf{Errore assoluto} & \textbf{Errore relativo}\\
        \hline
        1 & 0.5 & 3 & 1.64583 & 1.64872 & 0.00289 & 0.00175 \\
        \hline
        1 & 0.5 & 10 & 1.64872 & 1.64872 & 0 & 0 \\
        \hline
        1 & 0.5 & 50 & 1.64872 & 1.64872 & 0 & 0 \\
        \hline 
        1 & 0.5 & 100 & 1.64872 & 1.64872 & 0 & 0 \\
        \hline
        1 & 0.5 & 150 & 1.64872 & 1.64872 & 0 & 0 \\
        \hline
        1 & 30 & 3 & 4981 & $1.06865\cdot10^{13}$ & $1.06865\cdot10^{13}$ & 0.99999 \\
        \hline
        1 & 30 & 10 & $2.3883\cdot10^{8}$ & $1.06865\cdot10^{13}$ & $1.06863\cdot10^{13}$ & 0.99998 \\
        \hline
        1 & 30 & 50 & $1.06833\cdot10^{13}$ & $1.06865\cdot10^{13}$ & $3.2\cdot10^{9}$ & $2.99443\cdot10^{-4}$ \\
        \hline
        1 & 30 & 100 & $1.06865\cdot10^{13}$ & $1.06865\cdot10^{13}$ & 0 & 0 \\
        \hline
        1 & 30 & 150 & $1.06865\cdot10^{13}$ & $1.06865\cdot10^{13}$ & 0 & 0 \\
        \hline
        1 & -0.5 & 3 & 0.604167 & 0.606531 & 0.00236 & 0.00390 \\
        \hline
        1 & -0.5 & 10 & 0.606531 & 0.606531 & 0 & 0 \\
        \hline
        1 & -0.5 & 50 & 0.606531 & 0.606531 & 0 & 0 \\
        \hline
        1 & -0.5 & 100 & 0.606531 & 0.606531 & 0 & 0 \\
        \hline
        1 & -0.5 & 150 & 0.606531 & 0.606531 & 0 & 0 \\
        \hline
        1 & -30 & 3 & -4079 & $9.35762\cdot10^{-14}$ & 4079 & $4.35901\cdot10^{16}$ \\
        \hline
        1 & -30 & 10 & $1.21255\cdot10^{8}$ & $9.35762\cdot10^{-14}$ & 121255000 & $1.29579\cdot10^{21}$ \\
        \hline
        1 & -30 & 50 & $8.78229\cdot10^{8}$ & $9.35762\cdot10^{-14}$ & 878229000 & $9.38517\cdot10^{21}$ \\
        \hline
        1 & -30 & 100 & $-3.42134\cdot10^{-5}$ & $9.35762\cdot10^{-14}$ & $3.42134\cdot10^{-5}$ & 365620746.4\\
        \hline
        1 & -30 & 150 & $-3.42134\cdot10^{-5}$ & $9.35762\cdot10^{-14}$ & $3.42134\cdot10^{-5}$ & 365620746.4 \\
        \hline
        2 & -0.5 & 3 & 0.607595 & 1.64872 & 1.04113 & 0.63147 \\
        \hline
        2 & -0.5 & 10 & 0.606531 & 1.64872 & 1.04219 & 0.63212 \\
        \hline
        2 & -0.5 & 50 & 0.606531 & 1.64872 & 1.04219 & 0.63212 \\
        \hline
        2 & -0.5 & 100 & 0.606531 & 1.64872 & 1.04219 & 0.63212 \\
        \hline
        2 & -0.5 & 150 & 0.606531 & 1.64872 & 1.04219 & 0.63212 \\
        \hline
        2 & -30 & 3 & 0.000201 & $1.06865\cdot10^{13}$ & $1.06865\cdot10^{13}$ & 1 \\
        \hline
        2 & -30 & 10 & $1.31395\cdot10^{-8}$ & $1.06865\cdot10^{13}$ & $1.06865\cdot10^{13}$ & 1 \\
        \hline
        2 & -30 & 50 & $9.36248\cdot10^{-14}$ & $1.06865\cdot10^{13}$ & $1.06865\cdot10^{13}$ & 1 \\
        \hline
        2 & -30 & 100 & $9.35762\cdot10^{-14}$ & $1.06865\cdot10^{13}$ & $1.06865\cdot10^{13}$ & 1 \\
        \hline
        2 & -30 & 150 & $9.35762\cdot10^{-14}$ & $1.06865\cdot10^{13}$ & $1.06865\cdot10^{13}$ & 1 \\
        \hline
    \end{tabular}
\end{quote}
\newpage
\section{Terzo esercizio}
\begin{quote}
    Il più grande numero intero positivo $d$ tale che \begin{equation*}
        1+2^{-d} > 1
    \end{equation*}
    è:
    \begin{itemize}
        \item Doppia precisione: $d=53$
        \item Singola precisione: $d=24$
    \end{itemize}
\end{quote}
\end{document}