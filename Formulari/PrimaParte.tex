\documentclass[10pt]{article}
\usepackage{lscape}
\usepackage{geometry}
\usepackage{graphicx} % Necessario per scalare la tabella
\usepackage{adjustbox} % Opzionale, per un controllo più fine della dimensione
\usepackage{array}
\usepackage{amsmath}

\geometry{a4paper, left=0.5cm, right=1cm, top=1cm, bottom=1cm}

\pagestyle{empty}

\let\olditemize\itemize
\renewcommand\itemize{\olditemize\setlength\itemsep{0em}}

\begin{document}
\begin{landscape}

% Posizionamento della tabella in modo fisso e ridotto
\begin{picture}(0,0)
    \put(585,-535){ % Coordina verticalmente e orizzontalmente la posizione della tabella
        \scalebox{0.7}{ % Riduce la tabella al 70% della sua dimensione originale
        \begin{tabular}{|c c c c c|}
        \hline
        $\sqrt{1}=1$ & $\sqrt{4}=2$ & $\sqrt{9}=3$ & $\sqrt{16}=4$ & $\sqrt{25}=5$ \\
        $\sqrt{36}=6$ & $\sqrt{49}=7$ & $\sqrt{64}=8$ & $\sqrt{81}=9$ & $\sqrt{100}=10$\\
        $\sqrt{121}=11$ & $\sqrt{144}=12$ & $\sqrt{169}=13$ & $\sqrt{196}=14$ & $\sqrt{225}=15$ \\ 
        $\sqrt{256}=16$ & $\sqrt{289}=17$ & $\sqrt{324}=18$ & $\sqrt{361}=19$ & $\sqrt{400}=20$\\
        $\sqrt{441}=21$ & $\sqrt{484}=22$ & $\sqrt{529}=23$ & $\sqrt{576}=24$ & $\sqrt{625}=25$ \\ 
        $\sqrt{676}=26$ & $\sqrt{729}=27$ & $\sqrt{784}=28$ & $\sqrt{841}=29$ & $\sqrt{900}=30$\\
        \hline
        \end{tabular}
        }
    }
\end{picture}

% Testo normale qui
\noindent
\begin{minipage}[t]{0.49\textwidth}
Qui ci andranno gli esercizi già fatti
\end{minipage}
%\hfill
\begin{minipage}[t]{0.49\textwidth}
    \footnotesize
\begin{tabular}{| m{2cm} | m{15cm} |}
    \iffalse
    \hline
    Matrice Invertibili $A$ & 
    \begin{itemize} 
        \item $\det A \neq 0$ 
        \item $\det (A^{-1})=\frac{1}{\det A}$
        \item $A$ non invertibile se $A^{N}=0$
        \item Il prodotto di due matrici diagonali è diagonale
        \item Una matrice diagonale non è per forza invertibile (potrebbe avere degli zeri nella diagonale)
        \item Teorema di Binét: $\det(AB)=\det A \cdot \det B$
        \item Ogni matrice diagonale è simmetrica
    \end{itemize} 
    \\
    \hline
    Sistemi lineari & \begin{itemize}
        \item Rouché-Capelli: $\infty^{\#incognite-\text{rk}(A)}$ con \#incognite $\neq$ rk$(A)$
        \item Se \#incognite = rk$(A)$ allora esiste una sola soluzione (\#incognite sono le colonne, le soluzioni sono le righe)
        \item Gauss: $R_{i}=R_{i}+(\frac{-a_{i1}}{a_{jj}})R_{j}$
    \end{itemize} \\
    \hline
    Vettori & \begin{itemize}
        \item $\begin{bmatrix}
            x_{1} \\ x_{2}
        \end{bmatrix}+\begin{bmatrix}
            w_{1} \\ w_{2}
        \end{bmatrix} = \begin{bmatrix}
            x_{1}+w_{1} \\
            x_{2}+w_{2} \\
        \end{bmatrix}$
        \item Dipendenza lineare: $\alpha v_{1}+\beta v_{2} = 0$
        \item Indipendenza lineare: $\alpha v_{1}+\beta v_{2} = 0 \rightarrow \alpha = \beta = 0$
        \item I vettori sono base di $R^{N}$ quando la matrice composta dai vettori ha rango $N$
        \item $v_{1} \in \langle v_{2},v_{3}\rangle \rightarrow v_{1}=\alpha v_{2}+\beta v_{3}$
    \end{itemize} \\
    \hline
    Vario & 
    \begin{itemize}
        \item  
    \end{itemize} \\
    \hline
    \fi
    \hline
    Esercizio 1 & Fare Gauss per il rango, creare il sistema (prendo le x in comune e le tratto come libere), isolo le x, sostituisco le x trovate nel vettore X, eseguo la moltiplicazione con v e poi pongo a 0 il risultato, isolo una x, sostituisco nuovamente e poi costruisco il vettore prendendo i coefficienti\\
    \hline
\end{tabular}
\end{minipage}

\end{landscape}
\end{document}
